% LaTeX file for resume 
% This file uses the resume document class (res.cls)

\documentclass{res}
%\usepackage[margin=1.5in]{geometry}
%\topmargin=-0.5in
%\oddsidemargin -.5in
%\evensidemargin -.5in
%\textwidth=6.0in
%\itemsep=0in
%\parsep=0in


%\usepackage[margin=1.0in]{geometry}
%\usepackage{helvetica} % uses helvetica postscript font (download helvetica.sty)
%\usepackage{newcent}   % uses new century schoolbook postscript font
\usepackage{color} 
\usepackage{xspace}
\usepackage{hyperref}
\setlength{\textheight}{9.5in} % increase text height to fit on 1-page 
\newcommand{\latex}{\LaTeX\xspace}

\begin{document} 

 \moveleft.5\hoffset\centerline{\LARGE\bf Hirak Sarkar}
% \name{HIRAK SARKAR\\[12pt]}     % the \\[12pt] adds a blank
				        % line after name  
\moveleft\hoffset\vbox{\hrule width\resumewidth height 1pt}\smallskip    

\address{8 Acorn Ln\\Stony Brook, NY-11790\\Mob No.+1 6315208131}
\address{ email: \href{mailto:hsarkar@cs.stonybrook.edu}{hsarkar@cs.stonybrook.edu}  \\ WWW: www.hiraksarkar.com}  
%Webpage:\color{blue}\url {http://kgec.academia.edu/HirakSarkar} }
                                  
\begin{resume}


\section{Education}          
 
 
 \vspace{-0.1in}	
   \begin{tabbing}
   \hspace{2.3in}\= \hspace{2.6in}\= \kill % set up two tab positions
     {\bf Ph.D in Computer Science} \>   \>  2014 - {\it present}
   \end{tabbing}  \vspace{-20pt}      % suppress blank line after tabbing
  % {\bf Bachelor of Technology(Computer Science and Engineering)}  \\
       Statistical Inference in Biological Data, {\it advisor: Prof. Rob Patro}  \\        
       Stony Brook University, NY     \\
 
 
 \vspace{-0.1in}	
   \begin{tabbing}
   \hspace{2.3in}\= \hspace{2.6in}\= \kill % set up two tab positions
     {\bf Masters of Technology (Computer Science)}  \>     \>2011-2013 
   \end{tabbing}  \vspace{-20pt}      % suppress blank line after tabbing
  % {\bf Bachelor of Technology(Computer Science and Engineering)}  \\        
       Indian Statistical Institute     \\
       $1^{st}$ Class (Hons.) 


\vspace{-0.1in}	
\begin{tabbing}
\hspace{2.3in}\= \hspace{2.6in}\= \kill % set up two tab positions
{\bf Bachelor of Technology (Computer Science and Engineering)}  \>     \>2007-2011
\end{tabbing}  \vspace{-20pt}      % suppress blank line after tabbing
  % {\bf Bachelor of Technology(Computer Science and Engineering)}  \\        
West Bengal University of Technology     \\       
GPA: 8.88/10      \\   
%B.Tech Thesis Topic: {\bf \color{blue} \underline {GameSAT: A Structured Approach to Combine SLS SAT Solvers}}  \\



\section{Publications}
\begin{itemize}
\item{\color{blue}Fast, Lightweight Clustering of de novo Transcriptomes using Fragment Equivalence Classes} A Srivastava*, Hirak Sarkar*, Laraib Malik and Rob Patro (* \textit{Joint first authors}) \textit{\textbf{RECOMB-seq'16}}.
\item{\color{blue}RapMap: A Rapid, Sensitive and Accurate Tool for Mapping RNA-seq Reads to Transcriptomes} A Srivastava, Hirak Sarkar, Nitish Gupta and Rob Patro  \textit{\textbf{ISMB'16}}.
\item {\color{blue}``Voronoi Game on Graphs"} (Extended version) S. Bandyapadhyay, A. Banik, S. Das and H. Sarkar (in alphabetical order of surnames) {\it Journal of Theoretical Computer Science} 2015.
\item {\color{blue}``Voronoi Game on Graphs''} .  S. Bandyapadhyay, A. Banik, S. Das and H. Sarkar (in alphabetical order of surnames) Seventh International Workshop on Algorithms and Computation (WALCOM), February 2013.
\end{itemize}

\section{Programming Skills}
C,C++,Python

\section{\bf Open Source Tools Used}
Dendropy, BioNet  (Comp Bio) \\
NLTK, Scrapy, Scikit-learn, Stanford Parser, Pandas (Data Science) \\



\section{Relevant Coursework} 
\begin{itemize}
\item Artificial Intelligence,  Computational Biology, Analysis of Algorithms, Fundamental of Networks. (at {\it SBU})
\item Machine Learning \& Pattern Recognition, Image Processing, Stochastic Process, Optimization Algorithms, Computer Graphics. (at {\it
Indian Statistical Institute})
\end{itemize}



\section{Relevant Course Projects}
\begin{itemize}
 \item{ \it{IPID Header Survey:}} We used IPID headers to estimate the load over different servers, sampled from alexa
 list. The main challenge of the project is to detect the wrapping pattern and navigate through the global vs local IPID counter. 
 We also looked at the temporal pattern of for the different regional websites which shows interesting correlation with possible 
 working load at the server end. \\
 {\it Instructor: Prof. Phillipa Gill}
 
% \item{\it{Improving phylogenetic tree reconstruction by using network interaction data:}} The phylogeny tree reflects the evolution 
%of different species. The tree can be constructed from individual gene sequences, which may lead to inaccurate trees. We used information
%from interaction network to improve the process of reconstruction.  
 
 \item {{\it Classification \& Clustering of satellite images:}} We benchmarked different classification algorithms to cluster different 
 kind of landmasses from infrared images taken by satellite. \\
 {\it Instructor: Prof. C A Murthy}
  
 \item {{\it Some Geometric and Combinatorial Properties of Binary Matrices Related to
Discrete Tomography:}} Here we are trying to decompose an image matrix into matrices
each having orthogonal convex polygon also known as Ferrer?s digraph. An operation could
regenerate the original image from these matrices. The methods can be applied to image and
data compression. ({\it Masters dissertation}) 
{\it Advisor: Prof. Bhargab B Bhattacharya \& Prof. Sandip Das}

 \item 
 {{\it GameSAT- A Structured Approach to Combine SLS SAT Solvers:}} Here we used several existing heuristic algorithms to mix up with each other in a customized probability to solve combinatorial hard problems encoded as SAT problems. We used UBCSAT framework for experimentation. \\
({\it B.Tech dissertation}) {\it Advisor: Ashiqur KhudaBukhsh, CMU}

\item{{\it \LaTeX\xspace to HTML converter:}} Implementation of \LaTeX\xspace to  HTML using LEX and YACC tool. {\it Instructor: Prof. Mandar Mitra}

\item {{\it InstAntroid- A Real-time Android Media Surfer:}} We have designed intelligent searching
  mechanism which could judge the taste of the user and instantly find out the most probable
  result. We also have designed instant search with it in the android platform. The whole
  algorithm is implemented on top of a music player application.
 {\it Advisor: Kousik Dasgupta}
 
 \item {{\it ISHARA- A Software Hardware Interface for Human Computer Interaction(HCI):}} We have used standard computer vision algorithms to track finger movements.OpenCV library is used with VC++ platform is used for implementation of the project.
 
 \end{itemize}
 
            
\section{Experience}
\begin{itemize}
\item {\it Junior Research Fellow} in Department of Computer Science \& Engineering at Indian Institute of Technology, Kharagpur (IIT) 
from July, 2013 to Sept, 2013. I was a member of Complex Network Engineering Group. I did TA-ship for Introductory Programming 
Course in that brief stint. 
\item {\it Visiting Researcher} at Advanced Computing \& Microelectronics Unit, Indian Statistical Institute from October, 2013 to December, 
2013. I worked on Computational Geometry and Graph Theory
\item  Teaching Assistant for CSE549 (Computational Biology), CSE219 (Game Programming)
\item {\it Summer Assistantship} with Prof. Rob Patro from May, 2015 to July, 2015. We worked on Alignment free network clustering algorithm. 
\item {\it Summer Assistantship} with Prof. Adam Seipel from May, 2016 to July, 2016, at Simons Center for Quantitative Biology, Cold Spring Harbor Lab. 
\end{itemize}

\section{Awards and Honors}
   \begin{itemize}
   \item Awarded {\it Special CS Chair Fellowship} ({\it of \$10000} ) from Stony Brook University.
   \item Awarded {\it Masters Scholarship} by Ministry of Statistics and Programme Implementation, Govt. of India. 
   \item Awarded {\it NUS Research Scholarship} from National University of Singapore. ({\it Discontinued after 6 months})
   \item Received {\color{blue} First Prize} for project ISHARA in INNOVA contest organized by Institute of Engineering and Management(IEM) Calcutta.
   \end{itemize}

\end{resume}
\end{document}
