% LaTeX file for resume 
% This file uses the resume document class (res.cls)

\documentclass{res} 
%\usepackage{helvetica} % uses helvetica postscript font (download helvetica.sty)
%\usepackage{newcent}   % uses new century schoolbook postscript font
\usepackage{color} 
\usepackage{hyperref}
\setlength{\textheight}{9.5in} % increase text height to fit on 1-page 

\begin{document} 

 \moveleft.5\hoffset\centerline{\LARGE\bf Hirak Sarkar}
% \name{HIRAK SARKAR\\[12pt]}     % the \\[12pt] adds a blank
				        % line after name  
\moveleft\hoffset\vbox{\hrule width\resumewidth height 1pt}\smallskip    

\address{8 Acorn Ln\\Stony Brook, NY-11790\\Mob No.+1 6315208131}
\address{E-mail: \color{blue}  hiraksarkar.cs@gmail.com \\}  
%Webpage:\color{blue}\url {http://kgec.academia.edu/HirakSarkar} }
                                  
\begin{resume}


\section{Education}          
 
 
 \vspace{-0.1in}	
   \begin{tabbing}
   \hspace{2.3in}\= \hspace{2.6in}\= \kill % set up two tab positions
     {\bf Ph.D in Computer Science}  \>     \>2019 (expected)
   \end{tabbing}  \vspace{-20pt}      % suppress blank line after tabbing
  % {\bf Bachelor of Technology(Computer Science and Engineering)}  \\        
       Stony Brook University     \\
 
 
 \vspace{-0.1in}	
   \begin{tabbing}
   \hspace{2.3in}\= \hspace{2.6in}\= \kill % set up two tab positions
     {\bf Masters of Technology (Computer Science)}  \>     \>2013 
   \end{tabbing}  \vspace{-20pt}      % suppress blank line after tabbing
  % {\bf Bachelor of Technology(Computer Science and Engineering)}  \\        
       Indian Statistical Institute     \\
       $1^{st}$ Class (Hons.) 


\vspace{-0.1in}	
\begin{tabbing}
\hspace{2.3in}\= \hspace{2.6in}\= \kill % set up two tab positions
{\bf Bachelor of Technology (Computer Science and Engineering)}  \>     \>July, 2011
\end{tabbing}  \vspace{-20pt}      % suppress blank line after tabbing
  % {\bf Bachelor of Technology(Computer Science and Engineering)}  \\        
Kalyani Government Engineering College \\
West Bengal University of Technology     \\       
GPA: 8.8/10      \\   
%B.Tech Thesis Topic: {\bf \color{blue} \underline {GameSAT: A Structured Approach to Combine SLS SAT Solvers}}  \\



\section{Publications}
\begin{itemize}
\item{\color{blue}RapMap: A Rapid, Sensitive and Accurate Tool for Mapping RNA-seq Reads to Transcriptomes} A Srivastava, Hirak Sarkar and Rob Patro.  bioRxiv:\url{http://dx.doi.org/10.1101/029652}
\item {\color{blue}``Voronoi Game on Graphs"} (Extended version) S. Bandyapadhyay, A. Banik, S. Das and H. Sarkar (in alphabetical order of surnames) {\it Journal of Theoretical Computer Science} 2015.
\item {\color{blue}``Voronoi Game on Graphs''} . Seventh International Workshop on Algorithms and Computation (WALCOM), February 2013.
\end{itemize}


\section{Project Works}
\begin{itemize}
 
 
 \item {\bf Some Geometric and Combinatorial Properties of Binary Matrices Related to
Discrete Tomography:}
 Here we are trying to decompose an image matrix into matrices
each having orthogonal convex polygon also known as Ferrer?s digraph. An operation could
regenerate the original image from these matrices. The methods can be applied to image and
data compression.  

  \item {\bf New Variations of facility location problems:} Voronoi game is geometric model of optimal
  facility location problem and an important topic of computational geometry. It is a game
  theoretic model of optimal coverage problem where two players want to place their facilities
  in a network in order to serve maximum users.
  
  \item {\bf InstAntroid- A Real-time Android Media Surfer:} We have designed intelligent searching
  mechanism which could judge the taste of the user and instantly find out the most probable
  result. We also have designed instant search with it in the android platform. The whole
  algorithm is implemented on top of a music player application.
  
  \item 
 {\bf GameSAT- A Structured Approach to Combine SLS SAT Solvers:} Here we used several existing heuristic algorithms to mix up with each other in a customized probability to solve combinatorial hard problems encoded as SAT problems.We used UBCSAT framework for experimentation.
  
 \item {\bf ISHARA- A Software Hardware Interface for Human Computer Interaction(HCI):} We have used standard computer vision algorithms to track finger movements.OpenCV library is used with VC++ platform is used for implementation of the project.Currently we are trying to incorporate it with projector to implement virtualization. 
 

\end{itemize}           


\section{Relevant Coursework } 
Computational Biology\\
Analysis of Algorithms\\
Data and File Structure\\
Pattern Recognition\\
Computer Graphics\\
 
\section{\bf Programming Languages Known}          
  C, C++, Python\\ 

\section{\bf Open Source Tools Used}
NLTK, Scrapy, Scikit-learn, Stanford Parser


\section{Training}
   \begin{itemize}

\item Summer internship at Telidos Inc. Area of work includes data collection from public forums
and trend prediction from text data. I used various open source NLP tools for the project.

\item     Reading Course on Design and Analysis of Algorithm 
    Special Interest:Computational Geometry 
    under guidance of Prof.Arijit Bishnu at
    Indian Statistical Institute 

\item    Traning on RDBMS concepts with Oracle in
    IBM Advanced Career Education \\    
\end{itemize}     



\section{Reference}

\begin{itemize}
\item Prof. Sandip Das\\
ACMU \\
Indian Statistical Institute\\
203, B. T. Road \\
Kolkata 700 108 India \\
Phone: +91 33 2575 3002 \\
Fax: +91 33 2577 3035 \\
sandipdas@isical.ac.in\\

%\item Prof. Bhargab B Bhattacharya \\
%ACMU \\
%Indian Statistical Institute \\
%203, B. T. Road \\
%Kolkata 700 108 India \\
%Phone: +91 33 2575 3002 \\
%Fax: +91 33 2577 3035 \\
%bhargab@isical.ac.in \\

\item Ashiqur KhudaBukhsh \\
Ph.D Student \\
School of Computer Science \\
Carnegie Mellon University \\
akhudabu@cs.cmu.edu \\

\end{itemize}
 
\section{Awards and Honors}
   \begin{itemize}
   \item Received {\color{blue} First Prize} for project ISHARA in INNOVA contest organized by Institute of Engineering and Management(IEM) Calcutta.

   \item Project ISHARA was nominated for iTech speech 2010 a contest organized by Cognizent Technology Solutions,India.

   \item Ranked 1374 among 136027 participants in GATE 2011 with score 648/1000.   

   \end{itemize}

\section{Extra Curricular Activities} 

\begin{itemize}
         
\item    Like to read and write Poems and Blogs 
    I have a blog where I maintain my poems 
    web-address:{\color{blue}\url {www.kabyakotha.blogspot.com} } 
\end{itemize} 
 
\end{resume}
\end{document}
